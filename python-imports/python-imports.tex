% Created 2016-06-26 Sun 23:09
\documentclass[11pt]{article}
\usepackage[utf8]{inputenc}
\usepackage[T1]{fontenc}
\usepackage{fixltx2e}
\usepackage{graphicx}
\usepackage{grffile}
\usepackage{longtable}
\usepackage{wrapfig}
\usepackage{rotating}
\usepackage[normalem]{ulem}
\usepackage{amsmath}
\usepackage{textcomp}
\usepackage{amssymb}
\usepackage{capt-of}
\usepackage{hyperref}
\usepackage{minted}
\date{\today}
\title{Module and Package}
\hypersetup{
 pdfauthor={章立},
 pdftitle={Module and Package},
 pdfkeywords={},
 pdfsubject={},
 pdfcreator={Emacs 24.5.1 (Org mode 8.3.4)}, 
 pdflang={English}}
\begin{document}

\maketitle
\tableofcontents

\section*{Modules}
\label{sec:orgheadline1}
\begin{itemize}
\item Any Python source file is a module
\end{itemize}
\begin{minted}[]{python}
# spam.py
def grok(x):
    ...

def blah(x):
    ...
\end{minted}

\begin{itemize}
\item You use import to execute and access it
\end{itemize}
\begin{minted}[]{python}
a = spam.grok('hello')

from spam import grok
a = grok('hello')
\end{minted}

\section*{Namespaces}
\label{sec:orgheadline2}
\begin{itemize}
\item Each module is its own isolated world
\end{itemize}
\begin{minted}[]{python}
# spam.py

x = 42

def blah():
    print(x)

# eggs.py

x = 37 

def foo():
    print(x)
\end{minted}
\begin{itemize}
\item These definitions of x are different
\item What happens in Module, stays in a module
\end{itemize}

\section*{Global Variables}
\label{sec:orgheadline3}
\begin{itemize}
\item Global variables bind inside the same module
\end{itemize}
\begin{minted}[]{python}
# spam.py

x = 42

def blah():
    print(X)
\end{minted}
\begin{itemize}
\item Fuctions record their definition environment
\end{itemize}
\begin{minted}[]{python}
>>> from spam import blah
>>> blah.__module__
'spam'
>>> blah.__globals__
{ 'x': 42, ...}
>>>
\end{minted}

\section*{Module Execution}
\label{sec:orgheadline4}
\begin{itemize}
\item When a module is imported, \texttt{all of the
  statements in the module execute} one after
another until the end of the file is reached
\item The contents of the  module namespace are of the
global names that are still defined at the end
of the execution process
\item If there are scripting statements that carry out
tasks in the global scope (printing, creating
files, etc.), yout will see them run on import
\end{itemize}

\section*{\texttt{from module import}}
\label{sec:orgheadline11}
\begin{itemize}
\item Lifts selected symbols out of a module \textbf{after
importing it} and  makes them available locally
\end{itemize}
\begin{minted}[]{python}
from math import sin, cos

def rectangular(r, theta):
    x = r * cos(theta)
    y = r * sin(theta)
    return x, y
\end{minted}
\begin{itemize}
\item Allows parts of module to be used without having
to type the module prefix
\end{itemize}

\subsection*{\texttt{from module import *}}
\label{sec:orgheadline5}
\begin{itemize}
\item Takes \textbf{all symbols} from a module and places
them into local scope
\end{itemize}
\begin{minted}[]{python}
from math import *

def rectangular(r, theta):
    y = x * cos(theta)
    y = r * sin(theta)
    return x, y
\end{minted}
\begin{itemize}
\item Sometimes useful
\item Usually considered bad style (try to avoid)
\end{itemize}

\subsection*{Commentary}
\label{sec:orgheadline6}
\begin{itemize}
\item Variations on import do not change the way that
modules work
\end{itemize}
\begin{minted}[]{python}
import math as m
from math import cos, sin
from math import *
...
\end{minted}
\begin{itemize}
\item import always executes the \textbf{entire} file
\item Modules are still isolated environments
\item These variations are just manipulating names
\end{itemize}

\subsection*{Module Names}
\label{sec:orgheadline7}
\begin{itemize}
\item File names have to follow the rules
\end{itemize}
\begin{minted}[]{python}
# good.py

...

# 2bad.py

...
\end{minted}
\begin{itemize}
\item Comment: This mistake comes up a lot when
teaching Python to newcomers
\item Must be a valid identifier name
\item Also: avoid non-ASCII characters
\end{itemize}

\subsection*{Naming Conventions}
\label{sec:orgheadline8}
\begin{itemize}
\item It is standard practice for package and module
names to be concise and lowercase
\item \texttt{foo.py}
\item \textbf{not} \texttt{MyFooModule.py}
\end{itemize}
\subsection*{Module Search Path}
\label{sec:orgheadline9}
\begin{itemize}
\item If a file isn't on the path, it won't import
\end{itemize}
\begin{minted}[]{python}
>>> import sys
>>> sys.path ['',
      '/usr/local/lib/python34.zip',
      '/usr/local/lib/python3.4',
      '/usr/local/lib/python3.4/plat-darwin',
      '/usr/local/lib/python3.4/lib-dynload',
      '/usr/local/lib/python3.4/site-packages']
\end{minted}
\begin{itemize}
\item Sometimes you might hack it\ldots{}although doing so
feels ``dirty''
\end{itemize}
\begin{minted}[]{python}
import sys
      sys.path.append("/project/foo/myfiles")
\end{minted}

\subsection*{Module Cashe}
\label{sec:orgheadline10}
\begin{itemize}
\item Modules only get loaded once
\item There's a cache behind the scenes
\end{itemize}
\begin{minted}[]{python}
>>> import spam
>>> import sys
>>> 'spam' in sys.modules
True
>>> sys.modules['spam']
<module 'spam' from 'spam.py'>
\end{minted}
\begin{description}
\item[{Consequence}] If you make a change to the source
and repeat the import, nothing happens (often
furstrating to newcomers)
\end{description}

\section*{Module Reloading}
\label{sec:orgheadline12}
\begin{itemize}
\item You can force-reload a module, but you're never
supposed to do it
\end{itemize}
\begin{minted}[]{python}
>>> from importlib import reload
>>> reload(spam)
<module 'spam' from 'spam.py'>
\end{minted}
\begin{itemize}
\item Apparently zombies are spawned if you do this
\item No, seriously
\item Don't do it
\end{itemize}

\section*{\texttt{\_\_main\_\_} check}
\label{sec:orgheadline13}
\begin{itemize}
\item If a file might run as a main program, do this
\end{itemize}
\begin{minted}[]{python}
# spam.py

...
if __name__ == '__main__':
    # Running as the main program
    ...
\end{minted}
\begin{itemize}
\item Such code won't run on library import
\end{itemize}
\begin{minted}[]{python}
import spam    # Main code doesn't execute
\end{minted}

\begin{minted}[]{sh}
zsh % python spam.py  # Main code executes
\end{minted}

\section*{Packages}
\label{sec:orgheadline16}
\begin{itemize}
\item For larger collectons of code, it is usually
desirable to organize modules into a hierarchy
\end{itemize}
\begin{verbatim}
|--spam/
   |-- foo.py
   |-- bar/
       |-- grok.py
   ...
\end{verbatim}
\begin{itemize}
\item To do it, you just add \uline{\uline{init}}.py files
\end{itemize}
\begin{verbatim}
|--spam/
   |-- init.py
   |-- foo.py
   |-- bar/
       |-- grok.py
   ...
\end{verbatim}

\subsection*{Using a Package}
\label{sec:orgheadline14}
\begin{itemize}
\item import works the same way, multiple levels
\end{itemize}
\begin{minted}[]{python}
import spam.foo
from spam.bar import grok
\end{minted}
\begin{itemize}
\item The \texttt{\_\_init\_\_.py} file import at each level
\item Apparently you can do things in those files
\item We'll get to that
\end{itemize}

\subsection*{Comments}
\label{sec:orgheadline15}
\begin{itemize}
\item At a simple level, there's not much to \texttt{import}
\item \ldots{}except for everything else
\end{itemize}
\end{document}
