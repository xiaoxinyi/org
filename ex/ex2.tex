% Created 2016-07-03 Sun 11:51
\documentclass[bigger]{beamer}
\usepackage[utf8]{inputenc}
\usepackage[T1]{fontenc}
\usepackage{fixltx2e}
\usepackage{graphicx}
\usepackage{grffile}
\usepackage{longtable}
\usepackage{wrapfig}
\usepackage{rotating}
\usepackage[normalem]{ulem}
\usepackage{amsmath}
\usepackage{textcomp}
\usepackage{amssymb}
\usepackage{capt-of}
\usepackage{hyperref}
\usepackage{minted}
\author{Eric S Fraga}
\date{2010-03-30 Tue}
\title{Writing Beamer presentations in org-mode}
\hypersetup{
 pdfauthor={Eric S Fraga},
 pdftitle={Writing Beamer presentations in org-mode},
 pdfkeywords={},
 pdfsubject={},
 pdfcreator={Emacs 24.5.1 (Org mode 8.3.4)}, 
 pdflang={English}}
\begin{document}

\maketitle
\tableofcontents


\section{Introduction}
\label{sec:orgheadline4}

\subsection{A simple slide}
\label{sec:orgheadline1}
This slide consists of some text with a number of bullet points:

\begin{itemize}
\item the first, very @important@, point!
\item the previous point shows the use of the special markup which
translates to the Beamer specific \emph{alert} command for highlighting
text.
\end{itemize}

The above list could be numbered or any other type of list and may
include sub-lists.


\subsection{A more complex slide}
\label{sec:orgheadline3}
This slide illustrates the use of Beamer blocks.  The following text,
with its own headline, is displayed in a block:
\subsubsection[Org mode increases productivity]{Org mode increases productivity\hfill{}\textsc{B\_theorem}}
\label{sec:orgheadline2}
\begin{itemize}
\item org mode means not having to remember \LaTeX{} commands.
\item it is based on ascii text which is inherently portable.
\item Emacs!
\end{itemize}

\hfill \(\qed\)
\end{document}
